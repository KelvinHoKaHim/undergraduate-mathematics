\chapter{Axiomatic Set Theory}
\section{Zermelo-Frankel Axioms}
\subsection{Axiom of Existence, Extensionality, and Restricted Comprehension}
\begin{rem}
    From this chapter and onward, we will no longer write proofs in pure symbols, for writing proof like this is extremely inconvenient and daunting for the readers and the authors. We will write proofs in natural language from now on. But the reader should under
\end{rem}
In previous chapters, we have already gone through Naive Set Theory. We have witnessed a careless use of set would yield gruesome contradictions. Therefore, until now, we have only used set very carefully. We wish to develop a system of axiomatic set theory that would put clear demarcation on what is definable and what is not. We have already constructed the ZFC model in the previous chapter. We will now enter such a model and develop of Mathematics in its entirety from here.\\

To start off, there will be nothing for us to work on if not a single set exist. Hence, the following axiom
\begin{ax}
    \textbf{(Axiom of Existence)} There is a set ($\vdash \exists x(x=x)$)
\end{ax}
As we have discussed in naive set theory, set equality should be determined extensionally, i.e. two sets are equal when they have the same elements. 
\begin{ax}
    \textbf{(Axiom of Extensionality)} Suppose $A,B$ are sets, then the following are equivalent
    \begin{itemize}
        \item $A = B$
        \item for any $x$, $x\in A$ if and only if $x\in B$.
    \end{itemize}
    ($\vdash \forall x\forall y (x=y\leftrightarrow \forall z(z\in x\leftrightarrow z\in y))$)
\end{ax}
By these two axiom along, it allows us to define subsets 
\begin{defn}
    (Subset relation) Let $A,B$ be sets, then $A$ is a \textbf{subset} of $B$, written $A\subseteq B$, if for any $x$, $x\in A$ implies $y\in B$.
\end{defn}
\begin{defn}
    (Strict subset) Let $A,B$ be sets, then $A$ is a \textbf{strict subset} of $B$, written $A\subset B$, if $A\subseteq B$ and $A\neq B$.
\end{defn}
\begin{prop}
    Let $A,B$ be sets, then the following are equivalent
    \begin{itemize}
        \item $A=B$
        \item $A\subseteq B$ and $B\subseteq A$
    \end{itemize}
\end{prop}
\begin{proof}
    ~ 
    \begin{itemize}
        \item ($\Rightarrow$) Suppose $A=B$. Then for any $x$, $x\in A$ if and only if $x\in B$. Then we have $A\subseteq B$ and $B\subseteq A$.
        \item ($\Leftarrow$) Suppose $A= B$, then for any $x$, we have 1). if $x\in A$ then $x\in B$ and 2.) if $x\in B$ then $x\in A$. Then we have $A\subseteq B$ and $B\subseteq A$.
    \end{itemize}
\end{proof}
\begin{rem}
    This proposition is extremely useful. Virtually all proof involving set equality uses this proposition.
\end{rem}
By this point, we already know that \textit{a set is the extension of a predicate}, expressed symbolically, $\set{x|\varphi x}$. And we have seen how this definition leads to Russell's paradox. Until now, we have been using set very carefully. But if we were to develop a reliable language for mathematics, we have to prevent Russell's paradox from ever arising, otherwise we could never be sure of the validity of our theorem.\\
What is really wrong with the "set of all set that does not contain itself"? For convenience, write $R = \set{x|x\not\in x}$. Now we know that existence of $R$ implies contradictions. So $R$ cannot really exist. The writing "$R$" itself is merely an empty name that denotes nothing. To prevent naming a non-existent set, a strategy is to include only those elements from an already existing set. 
\begin{ax}
    (Axiom schema of restricted comprehension) Let $\varphi$ be a predicate and $A$ be a set. Then there exist a set $B$ such that for any $x$, $x\in B$ if and only if $x\in A$ and $\varphi x$. ($\vdash \forall x\exists y\forall z(z\in y\leftrightarrow z\in x\wedge \varphi x)$)
\end{ax}
From this axiom, I am only picking out elements that are already pre-existing by themselves.  
\begin{rem}
    A curious point can be pointed out here. The natural language formulation of this axiom is rather odd. In predicate logic, quantifiers do NOT range across predicates, so the writing "Let $\varphi$ be a predicate" is not technically legal. In the formal language formulation of this axiom, we are not confronted with this issue, the "proposition" $\vdash \forall x\exists y\forall z(z\in y\leftrightarrow z\in x\wedge \varphi x)$ is really left as a schema, the readers can substitute $\varphi$ with any predicates they favour. Hence, the word "schema" in the name of the axiom.
\end{rem}
\begin{prop}
    Let $A$ be a set and $\varphi$ be a predicate. Then the set defined by specification is unique
\end{prop}
\begin{proof}
    Trivial.
\end{proof}
\begin{rem}
    For simplicity, we can write $B$ as $\set{x\in A|\phi x}$
\end{rem}
\begin{thm}
    There exist a set $E$, called an \textbf{empty set}, such that for any $x$, $x\not\in \empty$. 
\end{thm}
\begin{proof}
    By the axiom of existence, there exist a set $A$. Let $E = \set{x\in A|x\neq x}$, then for any $x$, since $x=x$, then it is not the case that $x\neq x$. So, $x\not\in E$.
\end{proof}
\begin{prop}
    Let $A$ be a set and $\varphi$ be a predicate, then $\set{x\in A|\varphi x}\subseteq A$.
\end{prop}
\begin{proof}
    Trivial.
\end{proof}
\begin{prop}
    The empty set is unique. 
\end{prop}
\begin{proof}
    Let $E,E'$ be empty sets. Then for any $x$, $x\not\in E$. This makes the proposition "for any $x$, if $x\in E$ then $x\in E'$" true. Therefore, we have $E\subseteq E'$. By arguing the same way for $E'$, we have $E'\subseteq E$. Hence, $E= E'$.
\end{proof}
\begin{rem}
    The empty set is denoted $\empty$.
\end{rem}
\begin{prop}
    For any set $A$, $\empty\subseteq A$.
\end{prop}
\begin{proof}
    Since $x\not\in\empty$ for any $x$, the proposition "for any $x$, if $x\in \empty$ then $x\in A$" is true. So $\empty\subseteq A$.
\end{proof}
\begin{defn}
    (Set intersection) Let $A,B$ be sets, the \textbf{intersection} of $A$ and $B$, written $A\cap B$, is the set $A\cap B:= \set{x\in A|x\in B}$
\end{defn}
\begin{prop}
    (Commutativity of set intersection) For any set $A,B$, $A\cap B = B\cap A$.
\end{prop}
\begin{proof}
    ~
    \begin{itemize}
        \item ($\subseteq$) For any $x$, suppose $x\in A\cap B$, then $x\in A$ and $x\in B$. Then $x\in B$ and $x\in A$. So, $x\in B\cap A$.
        \item ($\supseteq$) By a similar argument, we can deduce that for any $x\in B\cap A$, $x\in A\cap B$.
        Therefore, $A\cap B = B\cap A$.
    \end{itemize}
\end{proof}
\begin{prop}
    (Associativity of set intersection) For any set $A,B,C$, $(A\cap B)\cap C = A\cap (B \cap C)$
\end{prop}
\begin{proof}
    For any $x$, the following are equivalent 
    \begin{align*}
        & x\in (A\cap B) \cap C\\
        \Leftrightarrow & x\in A\cap B \text{ and } x\in C\\
        \Leftrightarrow & (x\in A\text{ and } x\in B) and \in C\\
        \Leftrightarrow & x\in A \text{ and }(x\in B\text{ and  }x\in C)\\
        \Leftrightarrow & x\in A \text{ and }(x\in B\cap C)\\
        \Leftrightarrow & x\in A\cap(B\cap C)
    \end{align*}
    So, $(A\cap B)\cap C = A\cap(B\cap C)$
\end{proof}
\begin{prop}
    For any set $A$, $A\cap A = A$
\end{prop}
\begin{proof}
    ~
    \begin{itemize}
        \item ($\subseteq$) For all $x$, suppose $x\in A\cap A$. Then $x\in A$ and $x\in A$ and hence $x\in A$. Therefore, $x\in A$.
        \item ($\supseteq$) For all $x\in A$, we have $x\in A$ and hence $x\in A$ and $x\in A$. So, $x\in A\cap A$ 
    \end{itemize}
\end{proof}
\begin{prop}
    For all set $A$, $A\cap \empty = \empty$
\end{prop}
\begin{proof}
    ~
    \begin{itemize}
        \item ($\subseteq$) for any $x\in A\cap \empty$, we have $x\in \empty$.
        \item ($\supseteq$) for any $x$. Since $x\not\in \empty$. Then it is not that case that $x\in A$ and $x\in \empty$. So, $x\not\in A\cap \empty$ and therefore if $x\in A\cap \empty$, then $x\in \empty$.
    \end{itemize}
\end{proof}
\begin{prop}
    For all set $A,B,C$, if $A\subseteq B$ and $A\subseteq C$, then $A\subseteq B\cap C$.
\end{prop}
\begin{proof}
    For any $x\in A$, since $A\subseteq B$ and $A\subseteq C$, we have $x\in B$ and $x\in C$. So, $x\in B\cap C$. 
\end{proof}
\begin{defn}
    (Disjoint) Let $A,B$ be sets. $A,B$ are \textbf{disjoint} if $A\cap B = \empty$
\end{defn}
\begin{defn}
    (Pairwise disjoint) Let $A$ be a set, $A$ is \textbf{pairwise disjoint} when for any $x,y\in A$, if $x\neq y$, then $x\cap y= \empty$
\end{defn}
\begin{defn}
    (Set complement) Let $A,B$ be sets. Then the \textbf{complement} of $A$ with respect of $B$, denoted $A\setminus B$, is the set $A\setminus B := \set{x\in A|x\not\in B}$
\end{defn}
\begin{defn}
    For any set $A$, $A\setminus \empty = A$
\end{defn}
\begin{proof}
    ~
    \begin{itemize}
        \item ($\subseteq$) For any $x$, if $x\in A\setminus \empty$, then $x\in A$ and $x\not\in \empty$. Then $x\in A$. Hence, $A\setminus \empty \subseteq A$.
        \item ($\supseteq$) For any $x$, if $x\in A$. By definition of empty set, $x\not\in A$. So, $x\in A$ and $x\not\in \empty$. Hence, $x\in A\setminus \empty$. Therefore, $A\subseteq A\setminus\empty$.
    \end{itemize}
    Therefore, $A\setminus \empty= A$.
\end{proof}
\begin{prop}
    For any set $A$, $A\setminus A = \empty$ 
\end{prop}
\begin{proof}
    For any $x$, $x\not\in\empty$ by definition and it is not the case that $x\in A$ and $x\not\in A$. So, $x\not\in\empty$ and $x\not\in A\setminus A$. Then if $x\in\empty$ then $x\in A\setminus A$, and if $x\in A\setminus A$ then $x\not\in \empty$. Therefore, $A\setminus A = \empty$.
\end{proof}
\begin{prop}
    For any set $A,B$, $A\setminus B \subseteq A$
\end{prop}
\begin{proof}
    Pick any $x\in A\setminus B$, then we have $x\in A$ and $x\not\in B$. In particular, we have $x\in A$ and hence $A\setminus B \subseteq B$.
\end{proof}
\subsection{Axiom of Pairing, Union, and Power Set} 
The readers may have complainted by now that the axiom schema of specification may be too restrictive. The only set we know exist for sure is the empty set. And the axiom of comprehension only allows us to create set from an existing set. As a result, we could not make much out from these. Hence, we rely on three axioms that create a larger set from pre-existing sets.\\
The \textit{Axiom of Pair} describe the existence of a doubleton, the \textit{Axiom of Union} for a generalised union, and the \textit{Axiom of Power Set} for the power set.
\begin{ax}
    \textbf{(Axiom of Pairing)} For all $x,y$, there is a set $A$ such that $x\in A$ and $y\in A$. ($\vdash \forall x\forall y\exists z (x\in z\vee y\in z))$
\end{ax}
\begin{ax}
    \textbf{(Axiom of Union)} Let $\mathscr A$ be a set, then there exists a set $B$ such that for any $A\in \mathscr A$, if $x\in A$, then $x\in B$. ($\vdash \forall x\exists y\forall z (z\in x\rightarrow z\in y)$)
\end{ax}
\begin{ax}
    \textbf{(Axiom of Power Set)} Suppose $A$ be a set, then there is a set $P$ such that if $B\subseteq A$ is a subset, then $B\in P$. ($\vdash \forall x\exists y\forall z (z\subseteq x\rightarrow z\in y)$)
\end{ax}
It is actually wrong to say these axioms captures the idea of doubletons, unions, and power sets exactly. Take the Axiom of Pairing as an example, let $x,y$ be two distinct object, then the set $\set{x,y,\square, \triangle,\text{Hello}}$ also satisfies the conditions stated in the axiom. $\square, \triangle$, and Hello are unwanted element that we would like to be excluded from our set. The set will be the desired doubleton only after we applied the Axiom Schema of Restricted Comprehension.
\begin{prop}
    ($Doubleton$) For any $x,y$, there is a unique set $A$, called a \textbf{doubleton} of $x,y$, such that for any $z$, $z\in A$ if and only if $z=x$ or $z=y$.
\end{prop}
\begin{proof}
    Let $x,y$ be objects, then by the Axiom of Pairing, there is an $A$ such that $x\in z$ or $y\in z$. Then let $D = \set{w\in z| w=z\text{ or }w=y}$. It follows immediately that $D$ is the set we after. The Axiom Schema of Restricted Comprehension ensures its uniqueness.
\end{proof}
\begin{prop}
    (Singleton) For any $x$, there is a unique set $S$ such that for any $y$, $y\in S$ if and only if $y= x$.
\end{prop}
\begin{proof}
    The set $\set{x,x}$ is our desired set.
\end{proof}
The Axiom of Union and the Axiom of Power Set also faces the same problem, we also need to use Axiom Schema of Restricted Specification on these two two axioms to obtain the desired sets. The proofs go very similarly as the one we just had. 
\begin{prop}
    (Power Set) For any $A$, there exist a unique $P$, called the \textbf{power set} of $P$, such that for any $B$, $B\in P$ if and only if $B\subseteq A$.
\end{prop}
\begin{proof}
    Omitted.
\end{proof}
\begin{prop}
    (Union) For any $\mathscr A$, there is a unique set $U$, called the \textbf{generalised union of }$\mathscr A$, such that for any $x$, $x\in U$ if and only if $x\in A$ for some $A\in \mathscr A$.
\end{prop}
\begin{proof}
    Omitted.
\end{proof}
\begin{prop}
    (Generalised Intersection) For any set $\mathscr A$, if $\mathscr A$ is non-empty, there is a set $I$ such that for any $x$, $x\in I$ if and only if $x\in A$ for any $A\in\mathscr A$. 
\end{prop}
\begin{rem}
    We write the generalised union of $\mathscr A$ as $\bigcup\mathscr A$ or $\displaystyle \bigcup_{A\in\mathscr A}A$, the generalised intersection of $\mathscr A$ as $\bigcap \mathscr A$ or $\displaystyle\bigcap_{A\in\mathscr A}A$ and the power set of $A$ as $\mc P(A)$. Moreover, if $\mathscr A= \set{A,B}$, we write $\bigcup \mathscr A = A\cup B = B\cup A$, and $\bigcap A\cap B = B\cap A$. We call $A\cup B$ and $A\cap B$ simply their \textbf{union} and \textbf{intersections} respectively.
\end{rem}
Here, we will prove some very trivial yet important properties regarding union, intersections, and power sets.
\prop{(Associativity of set union) For any set $A,B,C$, $(A\cup B)\cup C = A\cup(B\cup C)$}
\begin{proof}
    For any $x$, the following are equivalent 
    \begin{align*}
        & x\in (A\cup B)\cup C\\
        \Leftrightarrow & x\in (A\cup B)\text{ or }x\in C\\
        \Leftrightarrow & (x\in A\text{ or }x\in B)\text{ or }x\in C\\
        \Leftrightarrow & x\in A\text{ or }(x\in B\text{ or }x\in C)\\
        \Leftrightarrow & x\in A \text{ or }x\in B\cup C\\
        \Leftrightarrow & x\in A \cup(B\cup C)
    \end{align*}
    Therefore, $(A\cup B)\cup C = A\cup (B\cup C)$
\end{proof}
\begin{prop}
    {For any set $A,B$, $A\subseteq A\cup B$ and $B\subseteq A\cup B$}
\end{prop}
\begin{proof}
    For any set $x$, if $x\in A$, then $x\in A$ or $x\in B$. Then $x\in A\cup B$. So, $x\in A\cup B$.\\
    Similarly, $B\subseteq B\cup A$. Since $A\cup B = B\cup A$, we have $B\subseteq A\cup B$.
\end{proof}
\begin{prop}
    {For any set $A$, $A\cup\empty = A$}
\end{prop}
\begin{proof}
    ($\subseteq$) For any $x$, if $x\in A\cup \empty$, then $x\in A$ or $x\in \empty$. But since $x\in\empty$ by definition, $x\in A$. Therefore, $A\cup \empty\subseteq A$\\
    ($\supseteq$) For any $x$, if $x\in A$, then $x\in A$ or $x\in \empty$. Then $x\in A\cup \empty$. Then $A\subseteq A\cup\empty$\\
    Therefore, $A\cup\empty = A$
\end{proof}
\begin{prop}
    {For any set $A$, $A\cup A = A$}
\end{prop}
\begin{proof}
    For any $x$, the following are equivlent. 
    \begin{align*}
        & x\in A\\
        \Leftrightarrow & x\in A \text{ or }x\in A\\
        \Leftrightarrow & x\in A\cup A
    \end{align*}
    So, $A\cup A = A$
\end{proof}
\begin{prop}
    {For any set $A,B,C$, if $A\subseteq C$ and $B\subseteq C$, then $A\cup B\subseteq C$}\
\end{prop}
\begin{proof}
    Pick any $x\in A\cup B$, then $x\in A$ or $x\in B$. 
    \begin{itemize}
        \item If $x\in A$, since $A\subseteq C$, we have $x\in C$
        \item If $x\in B$, since $B\subseteq C$, we have $x\in C$
    \end{itemize}
    In any case, $x\in C$. So, $A\cup B\subseteq C$.
\end{proof}
\begin{prop}
    {(Distributivity between set intersection and set union) For any set $A,B,C$, $(A\cup C)\cap(B\cup C) = (A\cap B)\cup C$ and $(A\cap C)\cup (B\cap C) = (A\cup B) \cap C$}
\end{prop}
\begin{proof}
    For any $x$, the following are equivalent 
    \begin{align*}
        & x\in (A\cup C)\cap(B\cup C)\\
        \Leftrightarrow & x\in (A\cup C) \text{ and }x\in(B\cup C)\\
        \Leftrightarrow & (x\in A\text{ or }x\in C)\text{ and }(x\in B\text{ or }x\in C)\\
        \Leftrightarrow & (x\in A \text{ and }x\in B)\text{ or }x\in C\\
        \Leftrightarrow & (x\in A\cap B)\text{ or }x\in C\\
        \Leftrightarrow & x\in(A\cap B)\cup C
    \end{align*}
    \begin{align*}
        & x\in (A\cap C)\cup (B\cap C)\\
        \Leftrightarrow & x\in A\cap C \text{ or }x\in B\cap C\\
        \Leftrightarrow & (x\in A\text{ and }x\in C)\text{ or }(x\in B\text{ and }x\in C)\\
        \Leftrightarrow & (x\in A\text{ or }x\in B) \text{ and }x\in C\\
        \Leftrightarrow & x\in A\cup B \text{ and }x\in C\\
        \Leftrightarrow & x\in(A\cup B)\cap C
    \end{align*}
\end{proof}
\begin{cor}
    {For any set $A,B,C$, $A\cap (B\cup C) = (A\cap B)\cup  (A\cap C)$ and $A\cup(B\cap C) = (A\cup B)\cap (A\cup C)$}
\end{cor}
\begin{proof}
    \begin{gather*}
        A\cap (B\cup C) = (B\cup C)\cap A = (B\cap A)\cup (C\cap A) = (A\cap B)\cup (A\cap C)\\
        A\cup (B\cap C) = (B\cap C)\cup A = (B\cup A)\cap (C\cup A) = (A\cup B)\cap (A\cup C)
    \end{gather*}
\end{proof}
\begin{prop}
    {(Distributivity between set intersection and set complement) For any set $A,B,C$, $(A\cap B)\setminus C = (A\setminus C) \cap (B\setminus C)$}
\end{prop}
\begin{proof}
    For any $x$, the following are equivalent 
    \begin{align*}
        & x\in(A\cap B)\setminus C\\
        \Leftrightarrow & x\in A\cap B\text{ and }x\not\in C\\
        \Leftrightarrow & (x\in A\text{ and }x\in B)\text{ and }x\not\in C\\
        \Leftrightarrow & (x\in A\text{ and }x\in B)\text{ and }(x\not\in C\text{ and } x\not\in C)\\
        \Leftrightarrow & (x\in A\text{ and }x\not\in C)\text{ and }(x\in B\text{ and }x\not\in C)\\
        \Leftrightarrow & x\in A\setminus C \text{ and }x\in B\setminus C\\
        \Leftrightarrow & x\in (A\setminus C)\cap (B\setminus C)
    \end{align*}
    So, $(A\cap B)\setminus C = (A\setminus C)\cap (B\setminus C)$
\end{proof}
\begin{prop}
    {(Distributivity of set complement I) For any set $A,B,C$, $(A\cup B)\setminus C = (A\setminus C)\cup (B\setminus C)$ and $(A\cap B)\setminus C = (A\setminus C)\cup (A\setminus B)$}
\end{prop}
\begin{proof}
    For all $x$, the following are equivalent, 
    \begin{align*}
        & x\in(A\cup B)\setminus C\\
        \Leftrightarrow & x\in A\cup B \text{ and }x\not\in C\\
        \Leftrightarrow & (x\in A\text{ or }x\in B)\text{ and }x\not\in C\\
        \Leftrightarrow & (x\in A\a x\not\in C)\o (x\in B\a x\not\in C)\\
        \Leftrightarrow & x\in A\setminus C\o x\in B\setminus C\\
        \Leftrightarrow & x\in(A\setminus C)\cup(B\setminus C)
    \end{align*}
    So, $(A\cup B)\setminus C = (A\setminus C)\cup (B\setminus C)$
    \begin{align*}
        & x\in(A\cap B)\setminus C\\
        \Leftrightarrow & x\in(A\cap B)\a x\not\in C\\
        \Leftrightarrow & (x\in A\a x\in B)\a x\not\in C\\
        \Leftrightarrow & (x\in A \a x\in B)\a (x\not\in C \a x\not\in C)\\
        \Leftrightarrow & (x\in A\a x\not\in C)\a (x\in B\a x\not\in C)\\
        \Leftrightarrow & x\in A\setminus C \a x\in B\setminus C\\
        \Leftrightarrow & x\in (A\setminus C)\cap (B\setminus C)
    \end{align*}
    So, $(A\cap B)\setminus C = (A\setminus C)\cup (A\setminus B)$
\end{proof}
\prop{(Distributivity of set complement II/De Morgan's law)For any set $A,B,C$, $A\setminus(B\cap C) = (A\setminus B) \cup (A\setminus C)$ and $A\setminus (B\cup C) = (A\setminus B)\cap (A\setminus C)$}
\begin{proof}
    For any $x$, the following are equivalent,
    \begin{align*}
        & x\in A\setminus (B\cap C)\\
        \Leftrightarrow & x\in A \text{ and }x\not\in B\cap C\\
        \Leftrightarrow & x\in A \text{ and }(x\not\in B\text{ or }x\not\in C)\\
        \Leftrightarrow & (x\in A\text{ and }x\not\in B)\text{ or }(x\in A\text{ and }x\not\in C)\\
        \Leftrightarrow & x\in A\setminus B\text{ or }x\in A\setminus C\\
        \Leftrightarrow & x\in (A\setminus B)\cup(A\setminus C)
    \end{align*}
    So, $A\setminus(B\cap C) = (A\setminus B) \cup (A\setminus C)$\\
    \begin{align*}
        & x\in A\setminus (B\cup C)\\
        \Leftrightarrow & x\in A\text{ and }x\not\in B\cup C\\
        \Leftrightarrow & x\in A \text{ and }(x\not\in B\text{ and }x\not\in C)\\
        \Leftrightarrow & (x\in A \text{ and } x\in A)\text{ and }(x\not\in B\text{ and }x\not\in C)\\
        \Leftrightarrow & (x\in A\text{ and }x\not\in B)\text{ and }(x\in A\text{ and }x\not\in C)\\
        \Leftrightarrow & x\in A\setminus B\text{ and }x\in A\setminus C\\
        \Leftrightarrow & x\in(A\setminus B)\cap(A\setminus C)\\
    \end{align*}
    So, $A\setminus (B\cup C) = (A\setminus B)\cap (A\setminus C)$
\end{proof}
\begin{prop}
    For any $A\in\mathscr A$, $A\subseteq \bigcup\mathscr A$
\end{prop}
\begin{proof}
    For any $A\in\mathscr A$, for any $x\in A$, by the definition of generalised union, we have $x\in\mathscr A$, then $A\subseteq \bigcup \mathscr A$.
\end{proof}
\begin{prop}
    Let $\mathscr A$ be a set. Suppose $A\in\mathscr A$. Then $\bigcap\mathscr A\subseteq A$.
\end{prop}
\begin{proof}
    For any $x\in \bigcap\mathscr A$, we have $x\in B$ for any $B\in\mathscr A$. In particular, we have $x\in A$. 
\end{proof}
\begin{prop}
    For any $\mathscr A, \mathscr B$, if $\mathscr A\subseteq \mathscr B$, then $\bigcup\mathscr A\subseteq \bigcup\mathscr B$.
\end{prop}
\begin{proof}
    Let $\mathscr A\subseteq \mathscr B$. The for any $x\in \bigcup A$, there is some $A\in\bigcup \mathscr A$ such that $x\in A$. Since $\mathscr A\subseteq B$, then $A$ is also a subset of $B$, then $x\in \bigcup B$.
\end{proof}
\begin{prop}
    For any $\mathscr A, \mathscr B$, if $\mathscr A\subseteq \mathscr B$, then $\bigcap \mathscr B\subseteq \bigcap\mathscr A$
\end{prop}
\begin{proof}
    Let $x\in\bigcap\mathscr B$, then for any $A\in\mathscr A$, since $\mathscr A\subseteq \mathscr B$, we have $A\in\mathscr B$. By definition of intersection, we have $x\in A$. Therefore, by the definition of intersection again, $x\in \bigcap\mathscr A$.
\end{proof}
\begin{prop}
    For any set $A$, $\empty\in\mc P(A)$
\end{prop}
\begin{proof}
    Since $\empty\subseteq P$, $\empty\in \mc P(A)$
\end{proof}
\begin{prop}
    For any set $A$, $A\in\mc P(A)$.
\end{prop}
\begin{proof}
    Since $A\subseteq A$, therefore $A\in\mc P(A)$.
\end{proof}
\begin{prop}
    Let $A,B$ be sets. Then $A\subseteq B$ if and only if $\mc P(A)\subseteq \mc P(B)$.
\end{prop}
\begin{proof}
    ~
    \begin{itemize}
        \item $(\Rightarrow)$ 
        \item ($\Leftarrow$) For any $x\in A$, we have $\set{x}\in\mc P(A)$. Since $\mc P(A)\subseteq \mc P(B)$, $\set{x}\in\mc P(B)$ as well. Then $\set{x}\subseteq B$ and hence $x\in B$.
    \end{itemize}
\end{proof}