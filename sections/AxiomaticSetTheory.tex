\chapter{Axiomatic Set Theory}
\section{Zermelo-Frankel Set Theory}
In the previous chapter on Naive set theory we have already developed a minimal account of sets. Now that we have transferred our domain of discussion from the metalanguage to the object language, we will continue our discussion of it.\\
~\\
A question one might ask is : Why do we need axiomatisation? Why don't we just continue our discussion in natural language. Our preceding discussion on set theory rely on an "trial-and-error" method. We give out a first characterisation of a concept, find some problems with it, revise it with a better one. This give rise to a problem --- What if we missed something? What if we were in fact mistaken with our characterisation without knowing it? To prevent our fourth crisis in Mathematics, we would like to introduce an axiomatic system. The validity of all theorems in this system rely on a small set of axioms. Then we can direct all our attention to only those axioms instead of inspecting every construction we have made. Only an axiomatised system can give us hope in devloping a valid account of Mathematics. But reader should always bear in mind an axiomatised system only ensures the validity of the language, but the language itself is not the entirety of Mathematics.\\
~\\
In the ontology of Zermelo-Frankel Set Theory(ZFC), everything is a set. This means every object in ZFC is either a collection or an empty set. But Mathematical objects are not themselves sets, at least we do not expect them do. Sets are merely a useful and powerful way to \textit{represent} those objects. This is why we must not confuse the language with the subject of discourse. This point would require more discussion. But they are only possible once we have reached the construction of different kinds of number.\\

\newpage
We will define some notations that we already semantically know their meaning. We will reiterate them in symbolic logic.
\begin{ax}
    \textbf{(Zermelo-Frankel Axioms)}
    ~\\
    \begin{itemize}
        \item[\textbf{ZFC1}]\textbf{(Axiom of Existence)} A set exists.\\
        $\vdash \exists x(x=x)$
        \item[\textbf{ZFC2}]\textbf{(Axiom of Foundation)} All sets contains a member that is disjoint to itself\\
        $\vdash \forall x(\exists y(y\in x)\rightarrow \exists z(z\in x\wedge \lnot\exists w(w\in z\wedge w\in x)))$
        \item[\textbf{ZFC3}]\textbf{(Axiom of Extension)} Two sets are equal if and only if they have the same elements\\
        $\vdash \forall x\forall y(x=y\leftrightarrow \forall z(z\in x\leftrightarrow z\in y))$
        \item[\textbf{ZFC4}]\textbf{(Axiom Schema of Specification)} Given a predicate $\varphi$. Let $A$ be a set, then there is a set $B$ such that for all $x\in B$ if and only if $x\in A$ and $\varphi x$.\\
        $\vdash \forall x\exists y\forall z(z\in y\leftrightarrow z\in x\wedge \varphi x)$
        \item[\textbf{ZFC5}]\textbf{(Axiom of Pairing)} Let $x,y$ be objects. Then there is a set containing $x$ and $y$.\\
        $\vdash \forall x\forall y\exists z (x\in z\wedge y\in z)$
        \item[\textbf{ZFC6}]\textbf{(Axiom of Union)} For each collection, there is a set that contains all the members of each sets in the collection.\\
        $\vdash \forall x\exists y \forall z\forall w(w\in y\wedge y\in x\to w\in y)$ 
        \item[\textbf{ZFC7}]\textbf{(Axiom of Power)} There is a collection that contains all the subsets of a set.
        $\vdash \forall x\exists y\forall z (\forall w(w\in z\to w\in x))\to z\in y$
        \item[\textbf{ZFC8}]\textbf{(Axiom Schema of Replacement)}
        \item[\textbf{ZFC9}]\textbf{(Axiom of Infinity)} There is a set that contains the empty set and if $x$ is a member of the set, then $x\cup\set{x}$ is also a member of the set.\\
        $\vdash \exists x((\exists y(\forall z(z\not\in y))\wedge y\in x)\wedge \forall w(w\in x\to \exists u (\forall v(v\in u\leftrightarrow v\in w\vee v= w))\wedge u\in x))$
        \item[\textbf{ZFC10}]\textbf{(Axiom of Choice)} For every collection, there is a set that contains exactly one element from each set of that collection.\\
        $\vdash \forall x((\exists a (a\in x)\wedge \forall r\forall s(r\in x\wedge s\in x\wedge  r\neq s\to\forall t(t\in r\leftrightarrow t\not\in s)))\to \exists y\forall z(z\in x\to \exists w(w\in z\wedge w\in y\wedge (\forall \ell (\ell\in z\wedge \ell \in y\to \ell = w)))))$ 
    \end{itemize}
\end{ax}
\begin{rem}
    I know, the axioms are ugly. An unconditional pursuit for rigor would result in a loss of mathematical beauty and intuition. Please bear with me, we will return to more accessible plain English formulation in a section or two.
\end{rem}
\begin{rem}
    All the relevant proof and definition related to the Axiom of Extension, Axiom Schema of Specification, Axiom of Pairing, Axiom of Union, and Axiom of Power could be rewritten into symbolic logic and inserted here. But for the sake of brevity, we will only hereby prove the major results. 
\end{rem}
We had quite some discussion with Axiom of Extension, Axiom Schema of Specification, Axiom of Pairing, Axiom of Union, and Axiom of Power. But in addition to these, we have some new axioms as well. These axioms are all results proved to be necessary and unprovable from the rest of ZFC. We will only discuss axiom of foundation here. We delay the discussion of Axiom Schema of Replacement, Axiom of Infinity, and Axiom of Choice until later.