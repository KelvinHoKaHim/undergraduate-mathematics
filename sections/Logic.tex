\chapter{Logic and Philosophy}
    \section{Formal and Informal Logic}
    There are two sorts of logic : \textit{informal logic} and \textit{formal logic}. Informal logic is the study of correct reasoning in our natural language. The kind of logic we are all familiar with. We use informal logic all the time. Even arguments as simple as 
    \begin{enumerate}
        \item All men are mortal
        \item Socrates is mortal 
        \item Ergo, Socrates is mortal
    \end{enumerate}
    Is a display of informal logic.\\
    Statement 1 and 2 are call the \textit{premises}, and 3 is the \textit{conclusion}.
    Notice that the argument feature above is \textit{entirely independent of its context}. Meaning if we replace the word "men", "mortal", "Socrates" with other things, the argument still stands nonetheless. For example
    \begin{enumerate}
        \item All cats are cute 
        \item Willard is a cat 
        \item Therefore, Willard is cute
    \end{enumerate}
    One would immediately recognise the two arguments are "the same". Not in terms of the meaning of each premise, but rather what made the first argument correct is quite clearly the same as the second argument. But not all arguments that follow the form "All M is P, S is M, so "S is P" would yield a true conclusion. Take, for example, that 
    \begin{enumerate}
        \item All animals are four-legged
        \item Chickens are animals 
        \item Chicken are four-legged
    \end{enumerate}
    the conclusion is wrong. What made the conclusion wrong is not in the form of the argument, but because we included a false premise in the argument. What made an argument "correct" requires two parts : 1.) We need the argument to have the right form. We call these arguments \textit{valid}. A valid argument ensures that if its premises are true, then its conclusion must be true; 2.) We need the argument to have true premises. Only valid arguments with true premises have true conclusions. These arguments we call them \textit{sound}.\\
    What I have said here remains pretty vague. What makes an argument valid? What makes statements true? We all have a vague idea of validity and true, but not a clear-cut \textit{definition} of them. We will clarify these ideas as we progress. We will not include a full-blown philosophical discourse. This is a book on mathematics afterall. But we will clarify these ideas "clear enough" so that we can safely use them in constructing our mathematical universe.\\
    Formal lo
    \section{Propositions, Facts, and Truth}